\documentclass{article}

% \usepackage{corl_2022} % Use this for the initial submission.
\usepackage[final]{corl_2022} % Uncomment for the camera-ready ``final'' version.
%\usepackage[preprint]{corl_2022} % Uncomment for pre-prints (e.g., arxiv); This is like ``final'', but will remove the CORL footnote.

\title{Formatting Instructions for CoRL 2022}


\author{%
	Ryan Burgert \\
	Department of Computer Science\\
	Stony Brook University\\
	Stony Brook, NY 11794 \\
	\texttt{rburgert@cs.stonybrook.edu} \\
	% \And
	% Michael Ryoo \\
	% Department of Computer Science\\
	% Stony Brook University\\
	% Stony Brook, NY 11794 \\
	% \texttt{mryoo@cs.stonybrook.edu} \\
	\And
	Emma Burgert \\
	Ryan Burgert's Dog\\
}


\begin{document}
\maketitle

%===============================================================================

\begin{abstract}
	% Abstracts should be a single paragraph, between 4--6 sentences long, ideally. Gross violations will trigger corrections at the camera-ready phase.

	% Because of the 6 senence limit, I've commented out some of the sentences...

	Unpaired image translation algorithms can be used for sim-to-real tasks, but most fail to generate temporally consistent results.
	% Because of this, the details or even identity of a particular object might change as it moves across a scene.
	We present a new approach that combines differential rendering with image translation to achieve temporal consistency over indefinite timescales.
%
	%TURNING TWO SENTENCES INTO ONE:
	% We call this algorithm TRITON (Texture Recovering Image Translation Network): an unsupervised, end-to-end, stateless sim-to-real algorithm.
	% Unlike previous techniques, TRITON leverages the underlying 3d geometry of input scenes by generating realistic-looking learnable neural textures.
	We call this algorithm TRITON (Texture Recovering Image Translation Network): an unsupervised, end-to-end, stateless sim-to-real algorithm that 
	leverages the underlying 3d geometry of input scenes by generating realistic-looking learnable neural textures.
%
	% These textures are projected onto the surfaces of input scenes, which are then fed through an image translation network to obtain the final outputs.
	By settling on a particular albedo for the objects in a scene, we ensure consistency between frames statelessly.
	Unlike previous algorithms, TRITON is not limited to camera movements - it can handle the movement of objects as well, making it useful for downstream tasks such as robotic manipulation.
	Our experiments show that in addition to achieving higher temporal consistency, the translations are closer to ground truth photographs than previous techniques.
\end{abstract}

% Two or three meaningful keywords should be added here
\keywords{CoRL, Robots, Learning} 

%===============================================================================

\section{Introduction}
	
    Submission to CoRL 2022 will be entirely electronic, via a web site (not email). Information about the submission process and \LaTeX{} templates are available on the conference web site at \url{http://www.robot-learning.org/}. For camera ready submission, use the \texttt{final} option for the \texttt{\textbackslash usepackage} command. 

%===============================================================================

\section{Citations}
\label{sec:citations}

	Citations can be made using either \textbackslash citep\{\} or \textbackslash citet\{\}, depending from the appropriateness. To avoid the citation moving to the next line, it is often a good practice to replace the space before with a tilde (\~{}) character.
	Example 1: ``CoRL is the best conference ever~\citep{Gauss1857}.''
	Example 2: ``\citet{Gauss1857} proved, both theoretically and numerically, that CoRL is the best conference ever.''
	
%===============================================================================


%===============================================================================


%===============================================================================

\clearpage
% The acknowledgments are automatically included only in the final and preprint versions of the paper.

%===============================================================================

% no \bibliographystyle is required, since the corl style is automatically used.
\bibliography{example}  % .bib

\end{document}
